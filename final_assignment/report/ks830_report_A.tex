\documentclass[10pt, twocolumn]{article}
%% Language and font encodings
\usepackage[british]{babel}

\usepackage[T1]{fontenc}

%% Sets page size and margins
\usepackage[a4paper,top=3cm,bottom=3cm,left=2cm,right=2cm]{geometry}

\usepackage{amsmath,amssymb}  % Better maths support & more symbols
\usepackage{bm}  % Define \bm{} to use bold math fonts
\usepackage{mathtools}

\usepackage[shortlabels]{enumitem}
\usepackage[normalem]{ulem}

\usepackage[utf8]{inputenc} % Any characters can be typed directly from the keyboard, eg éçñ
\DeclareUnicodeCharacter{2212}{-}

\usepackage{parskip}
\usepackage{graphicx}

\usepackage{hyperref}
\urlstyle{same}

% \renewcommand{\cfttoctitlefont}{\fontsize{12}{15}\selectfont\bfseries}
% \renewcommand\cftsecfont{\small}
% \renewcommand\cftsecafterpnum{\vskip 0pt}
% \renewcommand\cftsecpagefont{\small}

\usepackage{pdfsync}  % enable tex source and pdf output synchronicity


\usepackage{fancyhdr}
\fancypagestyle{first}{
	\fancyhf{} % clear all header and footers
	\renewcommand{\headrulewidth}{0pt} % remove the line
	\setlength{\footskip}{50pt}
	\fancyfoot[C]{\thepage}

}

\fancypagestyle{plain}{
	\fancyhf{} % clear all header and footers
	% \renewcommand{\headrulewidth}{0pt} % remove the line
	% \setlength{\headheight}{15pt}
	\setlength{\footskip}{50pt}
	\fancyfoot[C]{\thepage}
	\fancyhead[C]{\textbf{\svsubject—\svshortsubject{}\ (\texttt{ks830})}}
}

\def\svauthor{Kamilė Stankevičiūtė (\texttt{ks830})}
\def\college{Gonville \& Caius College}
\def\svsubject{Data Science: Principles and Practice}
\def\svshortsubject{Final Assignment A}

\usepackage{pdfpages}

\usepackage{minted}
\usemintedstyle{colorful}

\begin{document}
\thispagestyle{first}
\pagestyle{plain}
\twocolumn[{
\begin{center}
\LARGE
\textbf{Data Science: Principles and Practice \\ Final Assignment A} \\[4mm]

\large
Kamilė Stankevičiūtė (\texttt{ks830}) \\ Gonville \& Caius College

% \today % October 2019
\end{center} \vskip10mm}]


\section{Data exploration}

I examine the dataset of medical care records of diabetic patients over a period of 10 years (1999-2008), as presented in a given sample \textit{diabetic\_data\_balanced.csv}. 

\subsection{Preprocessing}

\paragraph{Anonymisation} Some patients have multiple encounter records (up to 15 per patient). This might skew the results of further analysis (whether exploration or classification) as models might learn to identify particular patients (through patient number or otherwise). For this reason at most one randomly sampled encounter per patient will be included in further analysis, and patient and encounter numbers removed. This leaves 7944 unique instances.

\paragraph{Missing values} The dataset has several features with missing values, most notably weight (97.0\%), payer code (97.6\%), and medical specialty (36.3\%). The first two will be excluded from further analysis. After comparing the numerical feature distributions for missing and non-missing medical specialty values, I will assume that the values are missing at random and replace them (as well as the other missing categorical feature values) with a separate category. There seems to be no numerical feature values missing.

\paragraph{Train and test split} I chose 90\%/10\% stratified train/test split with 7149 train and 795 test instances.

\subsection{Patient demographics}
\begin{figure}
	\centering
	\caption{Patient distribution by age.}
	\includegraphics[width=\linewidth]{age_count.png}\label{agecount}
\end{figure}

First I examine patient demographic data. Figure \ref{agecount} shows the positively-skewed patient age distribution, where the majority of patients are from 50 to 90 years old. Generally, more adult patients are readmitted than not, and we can observe that patients from the age of 70 have higher occurrence of `<30' label compared to `>30', indicating that they are more likely to be readmitted sooner.



\newpage
\clearpage


\end{document}