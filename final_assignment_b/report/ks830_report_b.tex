\documentclass[10pt, twocolumn]{article}
%% Language and font encodings
\usepackage[british]{babel}

\usepackage[T1]{fontenc}

%% Sets page size and margins
\usepackage[a4paper,top=3cm,bottom=3cm,left=2cm,right=2cm]{geometry}

\setlength{\columnsep}{12pt}

\usepackage{amsmath,amssymb}  % Better maths support & more symbols
\usepackage{bm}  % Define \bm{} to use bold math fonts
\usepackage{mathtools}

\usepackage[shortlabels]{enumitem}
\usepackage[normalem]{ulem}

\usepackage[utf8]{inputenc} % Any characters can be typed directly from the keyboard, eg éçñ
\DeclareUnicodeCharacter{2212}{-}

\usepackage{parskip}
\usepackage{graphicx}
\graphicspath{{figures/}}

\usepackage{subcaption}

\usepackage{tabularx}

\usepackage{hyperref}
\urlstyle{same}

% \renewcommand{\cfttoctitlefont}{\fontsize{12}{15}\selectfont\bfseries}
% \renewcommand\cftsecfont{\small}
% \renewcommand\cftsecafterpnum{\vskip 0pt}
% \renewcommand\cftsecpagefont{\small}

\usepackage{pdfsync}  % enable tex source and pdf output synchronicity


\usepackage{fancyhdr}
\fancypagestyle{first}{
	\fancyhf{} % clear all header and footers
	\renewcommand{\headrulewidth}{0pt} % remove the line
	\setlength{\footskip}{50pt}
	\fancyfoot[C]{\thepage}

}

\fancypagestyle{plain}{
	\fancyhf{} % clear all header and footers
	% \renewcommand{\headrulewidth}{0pt} % remove the line
	% \setlength{\headheight}{5pt}
	\setlength{\footskip}{50pt}
	\fancyfoot[C]{\thepage}
	\fancyhead[C]{\textbf{\svsubject—\svshortsubject{}\ (ks830)}}
}

\def\svauthor{Kamilė Stankevičiūtė (\texttt{ks830})}
\def\college{Gonville \& Caius College}
\def\svsubject{Data Science: Principles and Practice}
\def\svshortsubject{Final Assignment B}

\usepackage{pdfpages}
\usepackage{float}
\usepackage{stfloats}

% \usepackage{minted}
% \usemintedstyle{colorful}

\begin{document}

\thispagestyle{first}
\pagestyle{plain}
\twocolumn[{
\begin{center}
\LARGE
\textbf{Data Science: Principles and Practice \\ Final Assignment B} \\[4mm]

\large
Kamilė Stankevičiūtė (\texttt{ks830}) \\ Gonville \& Caius College \\[4mm]

? words
\end{center} \vskip10mm}]


\section{Data preparation}

I use the raw data (\textit{diabetic\_data\_original.csv}) of the same medical care record dataset \cite{strack2014dataset} as in Assignment A. The following section describes the changes in preprocessing steps with respect to the previous Assignment.

\subsection{Selection of training instances}
The preprocessing steps below resulted in 16515 training instances.

\paragraph{Filtering}
Due to the nature of the prediction task, only the patients with at least two recorded encounters were considered. For each patient, I did not use the last encounter as a datapoint because the ground truth (the length of next stay) for it is unknown: either the patient was not readmitted (we did not observe the counterfactual result of how long the stay would have been had the patient been readmitted), or the patient was readmitted but the ground truth for this has not been recorded. 

\paragraph{Multiple examples} To ensure independence of all the examples in the dataset, I only used a single randomly sampled example per patient. 

\paragraph{Multiple readmission} Some patients have recorded follow-on visits even in cases when the readmission outcome is `NO' (e.g. at least 44 patients have been \textit{not readmitted more than once}). Further investigation would be necessary to determine the reason for this in order to handle it correctly—I decided to exclude any no-readmission encounters still left after filtering and length-of-next-stay feature generation. 

\subsection{Feature preprocessing}
\paragraph{Anonymisation} As before, the encounter and patient numbers were removed for training.

\paragraph{Missing values} In the full dataset, the \textit{weight} feature was missing in 98.6\% of instances, and was not used for training. For other features, the missing values were imputed using median or constant strategy for numerical and categorical features respectively.

\paragraph{Numerical and categorical features} Similar to processing in Assignment A: the features were converted to numerical or categorical values based on their meaning. Diagnoses were first grouped into categories to avoid a large number of features with small number of examples. The numerical and categorical features were normalised and one-hot encoded respectively.

\paragraph{Feature sets} Since the dataset is larger and the neural network models more powerful, I trained on the full feature set (except weight).

\section{Machine learning set-up}

\subsection{Train and test set split}
With a view to get more generalisable model, I split the data into qualitatively different sets using a single-component t-SNE projection (at perplexity 30), holding out the top decile of values for testing. 


\subsection{Probabilistic model}

I model the predicted variable (length of next stay) using \textit{zero-truncated Poisson distribution} (ZTP). I truncated the zero because, if the follow-on visit happens, the length of stay must be at least 1, violating the assumptions of the regular Poisson distribution where observations of 0 are possible (on the other hand, I still allow prediction of more than 14 days). The observations $y_i \in \mathbb{N}^+$ are therefore modelled using \[Y_i \sim \mathrm{ZTP}(f_\theta(x_i))\]

where 

\begin{equation}
	\mathbb{P}[Y_i = y_i] = \frac{f_\theta(x_i)^{y_i}}{(e^{f_\theta(x_i)} - 1)y_i!}.
	\label{eq1}
\end{equation}

\paragraph{Loss function} For $M$ examples, parameters $\theta$ maximising the probability of the dataset (expressed as the product of probabilities in (\ref{eq1}) for each observation) minimise the following loss function:

\begin{equation}
	\mathcal{L} = \sum\limits_{i=1}^{M} -y_i \log(f_\theta(x_i)) + \log(e^{f_\theta(x_i)} - 1) + \log(y_i!)
\end{equation}

Following the TensorFlow documentation on Poisson loss function,\footnote{\url{https://www.tensorflow.org/api_docs/python/tf/nn/log_poisson_loss}} the training can be optimised computing the inexact version of this loss, omitting the constant $\log(y_i!)$ term.


\subsection{Neural network architecture}
TODO multiple architectures

TODO recurrent neural networks

TODO comparison on the number/width of hidden units 

TODO cross-validation for the unshuffled tSNE data? Select development sets by quantile. (although that does not necessarily give any \textit{more} insight to the \textit{reasons} behind the performance)

% You could just about squeeze in something, but it's not worth it --
% the real point of this practical is "how to I put together a neural
% network, and how does its design affect what it learns?"

I implemented a multilayer perceptron with two hidden layers of 64 and 32 units, with the sigmoid activation function as the non-linearity and used the Adam optimiser. 

\paragraph{Kernel initialisation} I used the kernel initialiser of \texttt{ones} which seemed to prevent the unstable initial state resulting in undefined floating point numbers, albeit with slightly slower convergence.


\paragraph{Hyperparameters} I used the default hyperparameters unless specified otherwise.

\section{Results}
TODO table and comparison of results, insights on which model is the best

TODO if results continue to be sh*t, then: current stay features are not enough to predict the future length of stay. Medical emergencies are unpredictable in nature. Perhaps patients with 


\subsection{Predictive power}

\paragraph{Interpretation}

\subsection{Impact of PCA on training performance}
TODO what are the \textit{components} of PCA? Are they more useful in combining and representing features? — Look at previous assignment component breakdown.

Does it speed training? \textbf{Why?}


\section{Evaluation}
TODO Evaluate the predictive power of the best model on the holdout set.


\medskip
 
\bibliographystyle{unsrt}
\bibliography{ks830_report_b}

\end{document}